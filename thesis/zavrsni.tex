\documentclass[times, utf8, zavrsni, numeric]{fer}
\usepackage{booktabs}
\usepackage{url}
\usepackage[hidelinks]{hyperref}

\begin{document}

% TODO: Navedite broj rada.
\thesisnumber{5173}

\title{Implementacija jezgre RTS igre}

\author{Leon Luttenberger}

\maketitle

% Ispis stranice s napomenom o umetanju izvornika rada. Uklonite naredbu \izvornik ako želite izbaciti tu stranicu.
\izvornik

% Dodavanje zahvale ili prazne stranice. Ako ne želite dodati zahvalu, naredbu ostavite radi prazne stranice.
\zahvala{}

\tableofcontents

\chapter{Uvod}
\par Strategije u stvarnom vremenu (engl. \textit{Real-Time Strategy}, u nastavku RTS) su žanr računalnih igara temeljene na skupljenju resursa za izgradnju jedinica i građevina s ciljem da se poraze protivnici. Za razliku od strategija na potez (engl. Turn-based strategy) gdje svaki igrač ima proizvoljno vrijeme za odigrati svoj potez, u RTS igrama svi igrači igraju svoje poteze u isto vrijeme znajući da svi protivnici rade isto. U širom smislu, RTS igra se može definirati kao simulacija bitke u stvarnom vremenu\cite{url:WhatDoesRTSMean}.

Odluke u RTS igrama se svode na dvije kategorije: mikro upravljanje i marko upravljanje\cite{article:HybridPathdinding}. Makro upravljanje predstavlja odluke na višoj razini abstrakcije, npr. koje jedinice i koje građevine gradit, kada započeti napad na protivnika, koje nadogradnje kupovat, itd. Mikro upravljanje se svodi na upravljanje samih jedinica, poput taktičkog pozicioniranja jedinica i uporabe njihovih specijalnih mogućnosti.



\chapter{Poglavlje 1}

\chapter{Poglavlje 2}

\chapter{Implementacija jezgrenih funkcionalnosti}

\chapter{Zaključak}
Zaključak.

\bibliography{literatura}
\bibliographystyle{fer} 

\begin{sazetak}
Računalne igre jedan su od vrlo značajnih pokretača razvoja računala. Postoji mnoštvo vrsta računalnih igara. Jednu od vrlo interesantnih vrsta igara čine strategije u stvarnom vremenu (engl. Real-Time Strategy, RTS). Razvoj takvih igara općenito uključuje pisanje mnogih podsustava i razvoj funkcionalnosti koje nisu specifične za konkretnu igru već ih je moguće višestruko iskorištavati. Takve funkcionalnosti moguće je izolirati u zaseban razvojni okvir. U okviru završnog rada potrebno je proučiti koji su sve elementi prisutni u RTS igri. Potrebno je napraviti programsku implementaciju osnovnih podsustava koje će biti moguće dijeliti između različitih RTS-igara. Također je potrebno implementirati prototip jedne konkretne igre koja sadrži osnovne elemente poput prikaza mape svijeta, stvaranja građevina različitih funkcija (proizvodnja, obrana) te jedinica, upravljanje jedinicama (grupiranje, zadavanje ciljeva: dolazak na zadani položaj, pucanje, prikupljanje resursa) i osnovno upravljanje protivničkim jedinicama. Implementaciju je potrebno ostvariti u programskom jeziku Java. Radu je potrebno priložiti algoritme, izvorne kodove i rezultate uz potrebna objašnjenja i dokumentaciju. Citirati korištenu literaturu i navesti dobivenu pomoć.

\kljucnerijeci{Ključne riječi, odvojene zarezima.}
\end{sazetak}

\engtitle{Implementation of RTS game core}
\begin{abstract}
Abstract.

\keywords{Keywords.}
\end{abstract}

\end{document}
