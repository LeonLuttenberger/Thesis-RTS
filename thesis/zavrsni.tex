\documentclass[times, utf8, zavrsni, numeric]{fer}
\usepackage{booktabs}
\usepackage{url}
\usepackage[hidelinks]{hyperref}

\begin{document}

% TODO: Navedite broj rada.
\thesisnumber{5173}

\title{Implementacija jezgre RTS igre}

\author{Leon Luttenberger}

\maketitle

% Ispis stranice s napomenom o umetanju izvornika rada. Uklonite naredbu \izvornik ako želite izbaciti tu stranicu.
\izvornik

% Dodavanje zahvale ili prazne stranice. Ako ne želite dodati zahvalu, naredbu ostavite radi prazne stranice.
\zahvala{}

\tableofcontents

\chapter{Uvod}
Uvod rada. Nakon uvoda dolaze poglavlja u kojima se obrađuje tema.

\chapter{Zaključak}
Zaključak.

\bibliography{literatura}
\bibliographystyle{fer}

\begin{sazetak}
Računalne igre jedan su od vrlo značajnih pokretača razvoja računala. Postoji mnoštvo vrsta računalnih igara. Jednu od vrlo interesantnih vrsta igara čine strategije u stvarnom vremenu (engl. Real-Time Strategy, RTS). Razvoj takvih igara općenito uključuje pisanje mnogih podsustava i razvoj funkcionalnosti koje nisu specifične za konkretnu igru već ih je moguće višestruko iskorištavati. Takve funkcionalnosti moguće je izolirati u zaseban razvojni okvir. U okviru završnog rada potrebno je proučiti koji su sve elementi prisutni u RTS igri. Potrebno je napraviti programsku implementaciju osnovnih podsustava koje će biti moguće dijeliti između različitih RTS-igara. Također je potrebno implementirati prototip jedne konkretne igre koja sadrži osnovne elemente poput prikaza mape svijeta, stvaranja građevina različitih funkcija (proizvodnja, obrana) te jedinica, upravljanje jedinicama (grupiranje, zadavanje ciljeva: dolazak na zadani položaj, pucanje, prikupljanje resursa) i osnovno upravljanje protivničkim jedinicama. Implementaciju je potrebno ostvariti u programskom jeziku Java. Radu je potrebno priložiti algoritme, izvorne kodove i rezultate uz potrebna objašnjenja i dokumentaciju. Citirati korištenu literaturu i navesti dobivenu pomoć.

\kljucnerijeci{Ključne riječi, odvojene zarezima.}
\end{sazetak}

\engtitle{Implementation of RTS game core}
\begin{abstract}
Abstract.

\keywords{Keywords.}
\end{abstract}

\end{document}
